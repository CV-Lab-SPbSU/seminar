\documentclass[notheorems,aspectratio=169]{beamer}
\usepackage{cmap}
\usepackage[T2A]{fontenc}
\usepackage[utf8]{inputenc}
\usepackage{mathtext}
\usepackage[english]{babel}
\usepackage{commath}
\usepackage{amsfonts}
\usepackage{amssymb}
\usepackage{amsmath}
\usepackage{amsthm}
\usepackage{mathtools}
\usepackage{indentfirst}
\usepackage{geometry}
\usepackage{tikz}
\usepackage{tkz-euclide}
\usetkzobj{all}
\usetikzlibrary{arrows,positioning}
\usetikzlibrary{shapes,snakes}
\usetikzlibrary{shapes.multipart}
\usepackage{graphicx}
\usepackage{epstopdf}
\usepackage{subcaption}
\usepackage{caption}
\usepackage{hyperref}
\usepackage{setspace}
\usepackage{float}
\usepackage{tcolorbox}
\usepackage{totcount}
\usepackage{xcntperchap}
\captionsetup{justification=centering}
\usepackage{pgfpages}
\usepackage{faktor}
\usepackage{listings}
\usepackage{bbold}
%\pgfpagesuselayout{4 on 1}[a4paper,border shrink=5mm, landscape]
\mode<presentation>
{
    \usetheme{Warsaw}
%    \usecolortheme{beaver}
    \setbeamercovered{transparent}
}

\regtotcounter{section}
\regtotcounter{subsection}
\RegisterCounters{section}{subsection}

\usepackage{color}

\definecolor{mygreen}{rgb}{0,0.6,0}
\definecolor{mygray}{rgb}{0.5,0.5,0.5}
\definecolor{mymauve}{rgb}{0.58,0,0.82}

\lstset{
  backgroundcolor=\color{white},
  basicstyle=\footnotesize,
  breakatwhitespace=false,
  breaklines=true,
  captionpos=b,
  commentstyle=\color{mygreen},
  deletekeywords={...},
  escapeinside={\%*}{*)},
  extendedchars=true,
  frame=single,
  keepspaces=true,
  keywordstyle=\color{blue},
  language=Octave,
  morekeywords={*,...},
  numbers=left,
  numbersep=5pt,
  numberstyle=\tiny\color{mygray}
  rulecolor=\color{black},
  showspaces=false,
  showstringspaces=false,
  showtabs=false,
  stepnumber=2,
  stringstyle=\color{mymauve},
  tabsize=2	                   %
%  title=\lstname
}


\makeatletter
% A new section definition to automatically set labels with the name: sec:sectionnumber
\let\oldsection\section
\renewcommand{\section}[1]{\oldsection{#1}\label{sec:\thesection}}
% A new subsection definition to automatically set labels with the name: sec:sectionnumber.subsectionnumer
\let\oldsubsection\subsection
\renewcommand{\subsection}[1]{\oldsubsection{#1}\label{sec:\thesection.\thesubsection}}
%init some calculators
\newcounter{calculator}
\newcounter{calcmaxsec}
\newcounter{calcmaxsubsec}
\providecommand{\TODO}{\fcolorbox{red}{red}{\large{TODO}}}
\newtheorem{theorem}{Theorem}
\newtheorem{example}{Example}
\newtheorem{definition}{Definition}
\DeclareMathOperator{\upd}{upd}
\DeclareMathOperator{\eol}{eol}
\DeclarePairedDelimiter\ceil{\lceil}{\rceil}
\DeclarePairedDelimiter\floor{\lfloor}{\rfloor}
\providecommand{\err}[1]{$\mathcal{#1}$}
\providecommand{\ev}[1]{$\mathbb{#1}$}
\providecommand{\TODO}[1]{\fcolorbox{red}{red}{\large{FIXME}:#1}}
\setbeamercovered{invisible}
\setbeamertemplate{navigation symbols}{}
\usepackage{physics}

\addtobeamertemplate{navigation symbols}{}{%
    \usebeamerfont{footline}%
    \usebeamercolor[fg]{footline}%
    \hspace{1em}%
    \insertframenumber/\inserttotalframenumber
}



\addtobeamertemplate{frametitle}{\setlength{\parindent}{0em}}{}
\addtobeamertemplate{block begin}{\setlength{\parindent}{0em}}{\setlength{\parindent}{2em}}
\addtobeamertemplate{block example begin}{\setlength{\parindent}{0em}}{\setlength{\parindent}{2em}}


